\documentclass[cn,10pt,math=newtx,citestyle=gb7714-2015,bibstyle=gb7714-2015]{elegantbook}

\title{从零开始编写C语言编译器}

\author{左元}
\date{May 2, 2021}

\extrainfo{最好的实践来自理论,最好的理论来自实践。—— 高德纳}

\setcounter{tocdepth}{3}

\logo{logo-blue.png}
\cover{cover.jpg}

% 本文档命令
\usepackage{array}
\newcommand{\ccr}[1]{\makecell{{\color{#1}\rule{1cm}{1cm}}}}

\usepackage{minted}
\usepackage{tcolorbox}
\BeforeBeginEnvironment{minted}{\begin{tcolorbox}[breakable]}%
\AfterEndEnvironment{minted}{\end{tcolorbox}}%

\definecolor{customcolor}{RGB}{32,178,170}
\colorlet{coverlinecolor}{customcolor}

\begin{document}

\maketitle
\frontmatter

\chapter*{特别声明}

\markboth{Introduction}{前言}

本书原著来自Github上的chibicc项目。

\begin{flushright}
左元 \\
May 2, 2021
\end{flushright}

\tableofcontents

\mainmatter

\chapter{本书简介}

\section{介绍}

在本书中,我们将创建一个程序,这个程序将C语言编写的源代码转换成x86-64汇编语言,编译器本身也是使用C语言开发的。我们的目标是编译器能够自举,也就是说我们写的编译器能够编译自己本身的源代码。

\chapter{第一步:编译整数}

先来编写main.c程序。

\begin{minted}{c}
    #include <stdio.h>
    #include <stdlib.h>

    int main(int argc, char **argv) {
        if (argc != 2) {
            fprintf(stderr, "%s: 无效的参数个数\n", argv[0]);
            return 1;
        }

        printf("  .globl main\n");
        printf("main:\n");
        printf("  mov $%d, %%rax\n", atoi(argv[1]));
        printf("  ret\n");
        return 0;
    }
\end{minted}

指令mov将整数移动到rax寄存器中,例如$\$1 \rightarrow \%rax$,就是将1这个整数移动到了rax寄存器中。

然后使用ret指令将main函数返回。

然后编译main.c文件。使用如下命令:

\begin{minted}{bash}
    cc -std=c11 -g -fno-common -c -o main.o main.c
    cc -o chibicc main.o
\end{minted}

编译出来的chibicc就是可执行程序。然后使用如下命令执行并查看结果:

\begin{minted}{bash}
    $ ./chibicc 233 > tmp.s
    $ gcc -static -o tmp tmp.s
    $ ./tmp
    $ echo $?
\end{minted}

然后会发现命令行出现了1这个整数,说明我们的编译器是成功的。

为了不每次这样编写命令行命令和编译命令,我们可以写一个Makefile和test.sh测试脚本来自动化我们的整个过程。

先来写test.sh测试脚本

\begin{minted}{bash}
    #!/bin/bash
    assert() {
        expected="$1"
        input="$2"

        ./chibicc "$input" > tmp.s || exit
        gcc -static -o tmp tmp.s
        ./tmp
        actual="$?"

        if [ "$actual" = "$expected" ]; then
            echo "$input => $actual"
        else
            echo "$input => $expected expected, but got $actual"
            exit 1
        fi
    }

    assert 0 0
    assert 42 42

    echo OK
\end{minted}

上面会检测输入和输出是否相等来检验我们的编译器是否写的正确。

接下来我们写Makefile文件。

\begin{minted}{makefile}
    CFLAGS=-std=c11 -g -fno-common

    chibicc: main.o
        $(CC) -o chibicc main.o $(LDFLAGS)

    test: chibicc
        ./test.sh

    clean:
        rm -f chibicc *.o *~ tmp*

    .PHONY: test clean
\end{minted}

\chapter{添加加减运算符}

此时我们的main.c变成了下面的样子:

\begin{minted}{c}
    #include <stdio.h>
    #include <stdlib.h>

    int main(int argc, char **argv) {
        if (argc != 2) {
            fprintf(stderr, "%s: 无效的参数个数\n", argv[0]);
            return 1;
        }

        char *p = argv[1]; // p指针指向输入的字符串,也就是第一个参数

        printf("  .globl main\n");
        printf("main:\n");
        // 现在输入变成了ld格式,长整形
        // strtol方法从p指向的指针开始向后寻找
        // 找出一个完整的十进制数值
        printf("  mov $%ld, %%rax\n", strtol(p, &p, 10));

        while (*p) {
            if (*p == '+') {
                p++; // 继续移动p指针
                // 继续寻找十进制数值
                // 将找到的数值和rax中的十进制数值进行相加
                // 然后保存在rax中
                // rax <= rax + num
                printf("  add $%ld, %%rax\n", strtol(p, &p, 10));
                continue;
            }

            if (*p == '-') {
                p++;
                printf("  sub $%ld, %%rax\n", strtol(p, &p, 10));
                continue;
            }

            fprintf(stderr, "未预期字符:'%c'\n", *p);
            return 1;
        }

        printf("  ret\n");
        return 0;
    }
\end{minted}

在test.sh中添加一条测试语句:

\begin{minted}[highlightlines={2}]{bash}
    assert 42 42
    assert 21 '5+20-4'

    echo OK
\end{minted}

\chapter{编写允许输入空白符的词法分析器}

此时main.c函数变成了如下的样子:

\begin{minted}[breaklines]{c}
    #include <ctype.h>
    #include <stdarg.h>
    #include <stdbool.h>
    #include <stdio.h>
    #include <stdlib.h>
    #include <string.h>

    // 记号的类型,用枚举来定义
    typedef enum {
        TK_PUNCT, // 分隔符,例如加减号就是分隔符
        TK_NUM,   // 数值类型
        TK_EOF,   // 文件结束符
    } TokenKind;

    // 记号的结构体定义
    typedef struct Token Token;
    struct Token {
        TokenKind kind; // 记号的类型
        Token *next;    // 下一个记号的指针
        int val;        // 如果是数值类型的话,它的值
        char *loc;      // 记号的位置
        int len;        // 记号的字符串长度
    };

    static void error(char *fmt, ...) {
        // va_list用于获取不确定个数的参数
        va_list ap;
        // va_start对va_list变量进行初始化,将ap指针指向参数列表中的第一个参数
        va_start(ap, fmt);
        // 从ap指针开始,将fmt可变参数列表打印出来,打印到标准错误文件描述符
        vfprintf(stderr, fmt, ap);
        fprintf(stderr, "\n");
        exit(1);
    }

    // 判断记号是否是某个运算符op
    static bool equal(Token *tok, char *op) {
        // 1. 比较存储区tok->loc和存储区op的前tok->len个字节是否相等,
        // 也就是指针从tok->loc和从op开始的字符串
        // 2. 判断op的最后一个字符是否是'\0'字符
        // 以上两个条件都必须满足
        return memcmp(tok->loc, op, tok->len) == 0 && op[tok->len] == '\0';
    }

    // 如果记号的字符串和s相等,则跳过
    static Token *skip(Token *tok, char *s) {
        if (!equal(tok, s))
            error("预期字符串是:'%s'", s);
        return tok->next;
    }

    // 获取记号中的数值的值
    static int get_number(Token *tok) {
        if (tok->kind != TK_NUM)
            error("预期是一个数值");
        return tok->val;
    }

    // 实例化一个记号
    static Token *new_token(TokenKind kind, char *start, char *end) {
        Token *tok = calloc(1, sizeof(Token)); // 分配一块大小为Token结构体的内存,用来存储记号
        tok->kind = kind;                      // 记号的类型
        tok->loc = start;                      // 记号的开始指针
        tok->len = end - start;                // 记号的长度
        return tok;                            // 将指针返回
    }

    // 将输入的字符串分割成一个一个的记号,使用链表的数据结构进行保存
    static Token *tokenize(char *p) {
        Token head = {};           // 空结构体
        Token *cur = &head;        // cur指针指向head结构体,或者说cur变量中保存了head结构体的地址

        // 当指针p不为空时,一直循环
        while (*p) {
            // 如果p指向的是空白字符,跳过
            if (isspace(*p)) {
                p++;
                continue;
            }

            // 如果p指向的是数字,那么实例化一个包含十进制整数的记号结构体
            if (isdigit(*p)) {
                // 先实例化一个数值记号结构体,然后将cur移动到下一个记号
                cur = cur->next = new_token(TK_NUM, p, p);
                // q指向p指向的地址
                char *q = p;
                // 从p指向的位置开始向后寻找一个无符号长整型数值
                // 然后将p指向后面第一个不是数字的位置
                cur->val = strtoul(p, &p, 10);
                // 计算整型记号的长度
                cur->len = p - q;
                continue;
            }

            // 如果p指向加减运算符,则实例化一个记号
            // 然后将cur指向下一个记号
            if (*p == '+' || *p == '-') {
                cur = cur->next = new_token(TK_PUNCT, p, p + 1);
                p++;
                continue;
            }

            error("无效的记号");
        }

        cur = cur->next = new_token(TK_EOF, p, p);
        return head.next;
    }

    int main(int argc, char **argv) {
        if (argc != 2)
            error("%s: 无效的参数个数", argv[0]);

        // 构建分割出来的记号的链表
        Token *tok = tokenize(argv[1]);

        // 样板代码
        printf(" .globl main\n");
        printf("main:\n");

        // 第一个记号必须是整型数值
        printf("  mov $%d, %%rax\n", get_number(tok));
        // 移动到下一个记号
        tok = tok->next;

        // 如果记号不是文件结束符,则一直循环
        while (tok->kind != TK_EOF) {
            // 如果记号是加号
            if (equal(tok, "+")) {
                printf("  add $%d, %%rax\n", get_number(tok->next));
                // 向后面移动两个记号
                tok = tok->next->next;
                continue;
            }

            // 如果记号是减号,则跳过记号
            tok = skip(tok, "-");
            printf("  sub $%d, %%rax\n", get_number(tok));
            tok = tok->next; // 向后移动一个记号
        }

        printf("  ret\n");
        return 0;
    }
\end{minted}

在test.sh中添加一条测试语句:

\begin{minted}[highlightlines={4}]{bash}
    assert 0 0
    assert 42 42
    assert 21 '5+20-4'
    assert 41 ' 12 + 34 - 5 '

    echo OK
\end{minted}

\chapter{改进一下错误信息}

以下是main.c程序:

\begin{minted}[breaklines]{c}
#include <ctype.h>
#include <stdarg.h>
#include <stdbool.h>
#include <stdio.h>
#include <stdlib.h>
#include <string.h>

typedef enum {
  TK_PUNCT, // 分隔符,比如加减运算符
  TK_NUM,   // 数值字面量
  TK_EOF,   // 文件结束符
} TokenKind;

// 记号的结构体
typedef struct Token Token;
struct Token {
  TokenKind kind; // 记号的类型
  Token *next;    // 下一个记号的指针
  int val;        // 如果是数值类型记号,则是数值的值
  char *loc;      // 记号的位置
  int len;        // 记号的字符串长度
};

// 输入字符串
static char *current_input;

// 报告错误然后退出
static void error(char *fmt, ...) {
  va_list ap;
  va_start(ap, fmt);
  vfprintf(stderr, fmt, ap);
  fprintf(stderr, "\n");
  exit(1);
}

// 报告错误的位置然后退出程序
static void verror_at(char *loc, char *fmt, va_list ap) {
  // 获取当前位置相对于输入开始指针的相对位置
  int pos = loc - current_input;
  fprintf(stderr, "%s\n", current_input);
  fprintf(stderr, "%*s", pos, ""); // 打印空白字符,一直打印到pos位置
  fprintf(stderr, "^ ");
  vfprintf(stderr, fmt, ap); // 打印错误
  fprintf(stderr, "\n");
  exit(1);
}

static void error_at(char *loc, char *fmt, ...) {
  va_list ap;
  va_start(ap, fmt);
  verror_at(loc, fmt, ap);
}

static void error_tok(Token *tok, char *fmt, ...) {
  va_list ap;
  va_start(ap, fmt);
  verror_at(tok->loc, fmt, ap);
}

// 判断当前记号和字符串s是否相等
static bool equal(Token *tok, char *op) {
  return memcmp(tok->loc, op, tok->len) == 0 && op[tok->len] == '\0';
}

// 跳过值为s的记号
static Token *skip(Token *tok, char *s) {
  if (!equal(tok, s))
    error_tok(tok, "expected '%s'", s);
  return tok->next;
}

// 返回数值记号中的数值
static int get_number(Token *tok) {
  if (tok->kind != TK_NUM)
    error_tok(tok, "expected a number");
  return tok->val;
}

// 实例化一个新的记号
static Token *new_token(TokenKind kind, char *start, char *end) {
  Token *tok = calloc(1, sizeof(Token));
  tok->kind = kind;
  tok->loc = start;
  tok->len = end - start;
  return tok;
}

// 对 `current_input` 进行词法分析,然后返回记号链表
static Token *tokenize(void) {
  char *p = current_input;
  Token head = {};
  Token *cur = &head;

  while (*p) {
    // 忽略空白符
    if (isspace(*p)) {
      p++;
      continue;
    }

    // 数值字面量
    if (isdigit(*p)) {
      cur = cur->next = new_token(TK_NUM, p, p);
      char *q = p;
      cur->val = strtoul(p, &p, 10);
      cur->len = p - q;
      continue;
    }

    // Punctuator
    if (*p == '+' || *p == '-') {
      cur = cur->next = new_token(TK_PUNCT, p, p + 1);
      p++;
      continue;
    }

    error_at(p, "invalid token");
  }

  cur = cur->next = new_token(TK_EOF, p, p);
  return head.next;
}

int main(int argc, char **argv) {
  if (argc != 2)
    error("%s: 无效参数个数", argv[0]);

  current_input = argv[1];
  Token *tok = tokenize();

  printf("  .globl main\n");
  printf("main:\n");

  // 第一个记号必须是一个数值
  printf("  mov $%d, %%rax\n", get_number(tok));
  tok = tok->next;

  // 后面跟着 `+ <number>` 或者 `- <number>`.
  while (tok->kind != TK_EOF) {
    if (equal(tok, "+")) {
      printf("  add $%d, %%rax\n", get_number(tok->next));
      tok = tok->next->next;
      continue;
    }

    tok = skip(tok, "-");
    printf("  sub $%d, %%rax\n", get_number(tok));
    tok = tok->next;
  }

  printf("  ret\n");
  return 0;
}
\end{minted}

\end{document}
